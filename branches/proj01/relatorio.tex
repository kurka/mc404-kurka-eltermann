
\documentclass{article}
\usepackage[brazil]{babel}
\usepackage[utf8]{inputenc}

\title{\textbf{MC404 - Trabalho 1}}
\author{David Burth Kurka (RA070589) \\
  Felipe Eltermann Braga (RA070803) \\
\date{\today}


\maketitle

\section{Introdução}\
Neste trabalho temos como objetivo, utilizando linguagem de montagem, ler um arquivo no formato .bmp em preto e branco e colorir de vermelho a borda das áreas da imagemm, gerando um novo arquivo de resposta.


\section{Especificacoes do programa}\
O programa é capaz de abrir um arquivo .bmp sem compressão com 24 bits por pixel e o tamanho máximo de 64Kb e contornar todas as regiões que conterem cores preto.
O arquivo do programa chama-se proj01.exe. Para executa-lo basta digitar proj01.exe e o nome de um arquivo bmp contido na mesma pasta do executavel.
Caso o arquivo não seja bmp (informação contida em seu header), ou o numero de bits por pixel for diferente de 24, ou o arquivo estiver compressado, uma mensagem de erro é exibida e o programa termina a execução.
Se o arquivo não apresenta esses problemas, le o arquivo, pinta os pixels de contorno de vermelho e cria um novo arquivo chamado saida.bmp na pasta onde está localizado o programa.

\section{Problemas}\

\end{document}
